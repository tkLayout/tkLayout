\documentclass[10pt,a4paper]{article}
\usepackage[latin1]{inputenc}
\usepackage[english]{babel}

\usepackage{hyphenat}
\usepackage{fancyhdr}
\pagestyle{fancy}

% Fancy header and footer to have my name everywhere
\lhead{\textbf{Multiple scattering}}
\chead{}
\rhead{\thepage /\pageref{LastPage}}
\lfoot{}
\cfoot{}
\rfoot{}
\renewcommand{\headrulewidth}{0.4pt}
%\renewcommand{\footrulewidth}{0.4pt}

% Page formatting (A4)
\pdfpagewidth 210mm
\pdfpageheight 297mm

% Margins and other stuff
% Vertical
\setlength\voffset{4.6mm}          % 1 in is automatically added (25.4mm)
\setlength\topmargin{0mm}
\setlength\headheight{0pt}
\setlength\headsep{24pt}
%\setlength\textheight{237mm}
\setlength\textheight{225mm}
\setlength\footskip{36pt}
% Horizontal
\setlength\hoffset{0mm}
\setlength\oddsidemargin{-12.9mm}  % 1 in is automatically added (25.4mm)
\setlength\evensidemargin{-12.9mm} % 1 in is automatically added (25.4mm)
\setlength\textwidth{18.5cm}
\setlength\headwidth{18.5cm}
\setlength\marginparsep{6pt}
\setlength\marginparwidth{0mm}
% Paragraph stuff
\setlength\parindent{0mm}
\setlength\parskip{10pt}

\begin{document}
Given the layer radii $x_n = x_1, x_2, \ldots, x_N$ with scattering angles $\theta_1, \theta_2, \ldots, \theta_3$, then
the deviation from the ideal path $y_n$ is
\begin{equation}
y_n=\sum_{i=1}^{n-1} \left (  x_n - x_i \right ) \theta_i
\end{equation}
The angles $\theta_i$ are distributed as a Gaussian, with r.m.s. such that
\begin{equation}
\left < \theta^2 \right > =
    \left ( \frac {13.6\,\mathrm{MeV}} {p} \right )^2
    \frac x {X_0}
    \left [ 1+ 0.038 \log \left ( \frac x {X_0} \right ) \right ] ^2
\end{equation}
The correlation between two deviations $y_n, y_m$ is (we will assume without loss of generality that $m \geq n$)
\begin{equation}
a_{n,m}\left < y_n y_m \right > =
      \left <
      \sum_{i=1}^{m-1} \left (  x_m - x_i \right ) \theta_i
      \times
      \sum_{i=1}^{n-1} \left (  x_n - x_i \right ) \theta_i
      \right >
\end{equation}
Since the angles $\theta_i$ are uncorrelated, any term containing in $\left < \theta_i \theta_j \right >$ with $i\neq j$ will be zero, thus
\begin{eqnarray}
a_{n,m}\left < y_n y_m \right > & = &
      \left <
      \sum_{i=1}^{m-1} \sum_{i=1}^{n-1}
      \left (  x_m - x_i \right )
      \left (  x_n - x_i \right ) \theta_i \theta_j \delta_{i,j}
      \right > \nonumber \\
      & = &
      \sum_{i=1}^{n-1} 
      \left (  x_m - x_i \right )
      \left (  x_n - x_i \right )  \left < \theta_i^2 \right > 
\end{eqnarray}
The measurement ``error'' depends both on the scattering of the real track with respect to the ideal case
and also on the intrinsic measurement error $\sigma_i$, which depends approximately on the strip pitch $p_i$
according to $\sigma_i = p_i / \sqrt{12}$, or $\sigma_i^2 = p_i^2 / 12$, thus the covariance matrix $b_{n,m}$ is
\begin{equation}
b_{n,m}= \left \{
\begin{array}{cl}
 \sum_{i=1}^{n-1} \left (  x_m - x_i \right ) \left (  x_n - x_i \right )  \left < \theta_i^2 \right > & n<m \\
 p_i^2 / 12 + \sum_{i=1}^{n-1} \left (  x_n - x_i \right )^2  \left < \theta_i^2 \right > & n=m \\
 c_{m,n}  & n>m
\end{array}
\right .
\end{equation}
Let's suppose we have $N$ hits, but in these $N$ only $M$ are measurement points and $N-M$ are hits
on inactive surfaces. In this matrix $b_{n,m}$ is computed exactly in the same way, but the rows
and columns corresponding to the inactive hits are removed. We thus start from a $N\times N$ square matrix
of correlations $b_{n,m}$ and we end up with a $M\times M$ measurement point covariance matrix $\c_{n,m}$,
or in matrix notation $C$.


\label{LastPage}
\end{document}
